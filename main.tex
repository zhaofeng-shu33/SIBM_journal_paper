\documentclass[journal]{IEEEtran}
\usepackage{amsthm}
\usepackage{amsmath}
\usepackage{amssymb}
\usepackage{bbm}
\newtheorem{theorem}{Theorem}
\newtheorem{definition}{Definition}
\newtheorem{lemma}{Lemma}
\newcommand{\ide}[2]{ \delta_{#1 #2} }
\DeclareMathOperator{\SSBM}{SSBM}
\DeclareMathOperator{\dist}{dist}

\title{Exact recovery of Stochastic Block Model with Ising sampling}
\author{
	Feng Zhao,~\IEEEmembership{Student Member, IEEE}\\
	Min Ye,~\IEEEmembership{Member, IEEE} and
	Shao-Lun~Huang,~\IEEEmembership{Member, IEEE}\\
	\thanks{Feng Zhao is with the
		Department of Electrical Engineering, Beijing, China.
		(Email: zhaof17@mails.tsinghua.edu.cn).
		Min Ye and S-L.~Huang are with the Data Science and Information
		Technology Research Center, Tsinghua-Berkeley Shenzhen Institute,
		Shenzhen, China (Email: \{yeemmi, shaolun.huang\}@sz.tsinghua.edu.cn).
	}}
	
\begin{document}
	\maketitle
\begin{abstract}
	Based on Ising sampling, we propose a stochastic algorithm to achieve the exact recovery for stochastic block model (SBM).
	The stochastic algorithm can be transformed to an optimization problem, which includes maximum likelihood and maximal modularity.
	Besides, we give an unbiased convergent estimator of the parameters of SBM, which can be computed in constant time.
	Finally, we use metropolis sampling to realize the theoretical Ising sampling and demonstrates the better performance of our method,
	compared with other theoretically guaranteed algorithm.
\end{abstract}
\begin{IEEEkeywords}
	stochastic block model, exact recovery, Ising model, modularity maximization, Metropolis sampling
\end{IEEEkeywords}
\section{Introduction}
Stochastic Block Model (SBM) is one of statistical modeling for community detection problems  \cite{holland1983stochastic, abbe2017community}.
It provides benchmark artificial dataset to evaluate different community detection algorithms.
Besides, SBM also inspires the design of algorithm for community detection tasks. These algorithms, such as
semi-definite relaxation, spectral clustering and label propagation, not only have theoretical guarantee when applied to SBM,
but perform well on dataset without SBM assumption. The study on the theoretical guarantee on SBM model can be divided between
exact recovery and partial recovery. For both cases, the asymptotic behavior of detection error
is analyzed when the scale of graph tends to infinity. There are already some well-known results of the exact recovery problem
on SBM.	To name but a few, Abbe and Mossel established the exact recovery region for a special sparse SBM with two communities  \cite{abbe2015exact, mossel2016}.
Later on, the result is extended to general SBM with multiple communities \cite{abbe2015community}.

Besides theoretical study on SBM, in community detection maximal modularity is a popular detection method \cite{Newman8577}.
The modularity is a kind of criteria and object function. Though maximal modularity works well in many practical problems, it's
generally unknown whether it can achieve exact recovery for SBM model. Newman, the inventor of modularity, demonstrates that
the modularity method is equivalent with maximum likelihood for a degree-corrected SBM model \cite{newman2016equivalence}. In this article,
we will fill the gap to prove that maximal modularity can achieve exact recovery for symmetric SBM model.

Our analysis of maximal modularity is based on Ising model, which is a probability distribution of node states \cite{ising1925beitrag}.
Ising model is originally proposed in statistical mechanics to model the ferromagnetism phenomenon but has widely application in neuroscience, information theory
and social networks. Among different variants of Ising models, the phase transition property is shared. Based on the random graph generated by SBM with two underlining communities,
the Ising model is first studied by \cite{ye2020exact}. Our work will extend the existing result to multiple community case and establish the phase transition
property and sample complexity results. Then we will propose a specialized Ising model using the definition of modularity. Sampling from this model,
we can also achieve exact recovery for SBM.

In order to achieve exact recovery of SBM, the extra parameters of Ising model should be carefully chosen, which depends on the parameters of SBM. 
Previous methods require jointly estimation of node labels and model parameters \cite{nowicki2001estimation}, which are not suitable when only model parameters are required.
In this article, we propose an unbiased convergent estimator for SBM parameters when the number of communities is given. This estimator is not only useful
for Ising sampling, but also beneficial for other recovery algorithms which requires SBM parameters.

Exact solution to maximize the modularity or exact sampling from modularity-based Ising model is NP hard. Many algorithms have been proposed to find an approximation of maximal modularity in polynomial time.
Among these algorithms, simulated annealing performs well and produces a solution very close to the true maximal value \cite{liu2010detecting}.
On the other hand, in original Ising model,
metropolis sequential sampling is used to generate samples for Ising model \cite{metropolis1953equation}. Simulated annealing and metropolis sampling are closely related. In this article, we will
use metropolis sampling technique to sample from Ising model on SBM. We will demonstrate by experiments that for non-exact recovery region of SBM our method outperforms other
theoretically guaranteed algorithm like SDP and spectral clustering.

This paper is organized as follows. Firstly, in section \ref{sec:sibm} we introduce the stochastic Ising block model for both two and multiple communities.
Then in the next section \ref{sec:dcsibm}, we propose Degree-corrected Stochastic Ising Block Model and uses this model to prove the exact recovery by modularity
maximization algorithm. Besides, in section \ref{sec:psbm}, we give a parameter estimator for SBM. Based on this estimator, in section \ref{sec:ms},
we propose a community detection method which use metropolis algorithm to generate sample for Ising model. Numerical experiments and conclusion are given at last
to finish this paper.

Throughout this paper, the number of community is denoted by $k$; $m$ is the number of samples; $\lfloor x \rfloor$ is the floor function of $x$; the random undirected graph $G$ is written as $G(V,E)$ with vertex set $V$ and edge set $E$;
$V=\{1,\dots, n\} =: [n]$;
the label of each node is a random variable $X_i$; $X_i$ is chosen from $W= \{1, \omega, \dots, \omega^{k-1}\}$ and we further require $W$
is a cyclic group with order $k$; $W^n$ is the n-ary Cartesian power of $W$; $f$ is a permutation function on $W$ and applied to $W^n$ in elementwise way; the set $S_k$ is used to represent all permutation functions on $W$ and $S_k(\sigma):=\{f(\sigma)| f\in S_k\}$ for $\sigma \in W^n$; the indicator function $\ide{x}{y}$ or $\delta(x,y)$ is defined as
$\ide{x}{y} = 1 $ when $x=y$, and $\ide{x}{y}=0$ when $x\neq y$; $g(n) = \Theta(f(n))$ if there exists constant $c_1 < c_2$ such that $c_1 f(n) \leq g(n) \leq c_2 f(n)$
for large $n$;
$\Lambda := \{ \omega^j  \cdot \mathbf{1}_n | j=0, \dots,k-1\}$
where $\mathbf{1}_n$ is the all one vector with dimension $n$;
we define the distance of two vectors as:
$\dist(\sigma, X)
=|\{i\in[n]:\sigma_i\neq X_i\}| \textrm{ for } \sigma,X\in W^n
$ and the distance of a vector to a space $S\subseteq W^n$
as
$\dist(\sigma,S)
:=\min\{\dist(\sigma, \sigma') | \sigma' \in S\}
$.

\section{Stochastic Ising Block Model}\label{sec:sibm}
We consider a special symmetric stochastic block model, which is defined as follows:
	\begin{definition}[SSBM with $k$ communities] \label{def:SSBM}
	Let $0\leq q<p\leq 1$ and $V=[n]$. The random vector $X=(X_1,\dots,X_n)\in W^n$ and random graph $G$ are drawn under $\SSBM(n,k,p,q)$ if
	\begin{enumerate}
		\item $X$ is drawn uniformly with the constraint that $|\{v \in [n] : X_v = u\}| = \frac{n}{k}$ for $u\in W$;
		
		\item There is an edge of $G$ between the vertices $i$ and $j$ with probability $p$ if $X_i=X_j$ and with probability $q$ if $X_i \neq X_j$; the existence of each edge is independent with each other.
	\end{enumerate}
\end{definition}
Under SSBM, we use $Z_{ij}:=\mathbbm{1}[\{i,j\} \in E(G)]$, which is the indicator function of the existence of an edge between node $i$ and $j$.
Given the node labels $X$, $Z_{ij}$ is a Bernoulli random variable, whose expectation is given by:
\begin{equation}
\mathbb{E}[Z_{ij}] =
\begin{cases}
p & X_i = X_j \\ 
q & X_i \neq X_j
\end{cases}
\end{equation}

\section{Degree-corrected Stochastic Ising Block Model}\label{sec:dcsibm}
\section{Parameter Estimation of SBM}\label{sec:psbm}
\section{Community Detection based on Metropolis Sampling}\label{sec:ms}
\section{Conclusion}
\bibliographystyle{IEEEtran}
\bibliography{exportlist}
\end{document}